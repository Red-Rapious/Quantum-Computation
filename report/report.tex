\documentclass[12pt,a4paper]{article} 
\usepackage[a4paper,margin=2cm]{geometry}

\usepackage{cmap}
\usepackage[T1]{fontenc}
\usepackage[utf8]{inputenc}
\usepackage[kerning=true]{microtype}
\usepackage{lmodern}
\usepackage{algorithm}
\usepackage{algpseudocode}

\usepackage{amsmath}
\usepackage{amsfonts}
\usepackage{amssymb}

\usepackage{tikz}

\usepackage{hyperref}

\makeatletter

\title{\vspace{-3ex} \textbf{Quantum Computation}}
\author{Antoine Groudiev}
\date{\vspace{-1ex}Last edited \today}
\markright{Antoine Groudiev}{}
\pagestyle{myheadings}

\begin{document}
\maketitle
\tableofcontents

\section*{Introduction to Quantum Computers}
\addcontentsline{toc}{section}{Introduction to Quantum Computers}

\section{Quantum Computational Models}
\subsection{Quantum Logic Gates}

\subsection{Quantum Turing Machine}

\section{Quantum Complexity Theory}
\subsection{Introduction}
\subsection{Relationship between classical and quantum complexity classes}
\subsubsection{Simulating a quantum computer}
\subsubsection{Efficiently simulating a classical computer}

\subsection{The BQP class}
\subsection{Query complexity}

\section{An example of Quantum Algorithm: Shor's Algorithm}
\subsection{Motivation and overview}
\subsection{Classical part}
\subsection{Quantum part}
\subsubsection{Quantum Fourier Transform}

\section*{Conclusion}
\addcontentsline{toc}{section}{Conclusion}

\end{document}