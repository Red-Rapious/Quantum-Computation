\documentclass[aspectratio=169]{beamer}
\usetheme{Singapore}

\usepackage{cmap}
\usepackage[T1]{fontenc}
\usepackage[french]{babel}

\AtBeginSection[]
{
  \begin{frame}
    \frametitle{Plan}
    \tableofcontents[currentsection]
  \end{frame}
}

\title{\textbf{Calcul et informatique quantique:\\une introduction formelle}}
\author{Antoine Groudiev}
\institute{ENS Ulm}
\date{Janvier 2024}

\begin{document}
\frame{\titlepage}

\begin{frame}
    \frametitle{Plan}
    \tableofcontents
\end{frame}

\section{Introduction à l'informatique quantique}
\begin{frame}
    \frametitle{Introduction}
\end{frame}

\subsection{Notation de Dirac}
\begin{frame}
    \frametitle{Notation de Dirac}
\end{frame}

\subsection{Représentation vectorielle}
\begin{frame}
    \frametitle{Représentation vectorielle}
\end{frame}

\subsection{Sphère de Bloch}
\begin{frame}
    \frametitle{Visualisation avec la sphère de Bloch}
\end{frame}

\section{Modèles de calculabilité quantique}
\subsection{Circuits quantique}
\begin{frame}
    \frametitle{Porte $X$}
\end{frame}

\begin{frame}
    \frametitle{Porte $Z$}
\end{frame}

\begin{frame}
    \frametitle{Porte de Hadamard}
\end{frame}

\begin{frame}
    \frametitle{Intrication quantique}
\end{frame}

\begin{frame}
    \frametitle{Porte $CNOT$}
\end{frame}

\subsection{Langages, automates, grammaires quantiques}
\begin{frame}
    \frametitle{Langage quantique}
\end{frame}

\begin{frame}
    \frametitle{Automate quantique fini}
\end{frame}

\begin{frame}
    \frametitle{Langage quantique régulier et propriétés}
\end{frame}

\begin{frame}
    \frametitle{Grammaire quantique}
\end{frame}

\begin{frame}
    \frametitle{Automate à pile quantique}
\end{frame}

\begin{frame}
    \frametitle{Machine de Turing quantique}
\end{frame}

\section{Théorie de la complexité quantique}
\subsection{Classe BQP}
\begin{frame}
    \frametitle{Classe BQP (Bounded-error Quantum Polynomial time)}
\end{frame}

\begin{frame}
    \frametitle{Un problème Promise-BQP-complet}
\end{frame}

\begin{frame}
    \frametitle{Positionnement par rapport aux classes de complexité classiques}
\end{frame}

\subsection{Thèse de Church-Turing}
\begin{frame}
    \frametitle{Thèse de Church-Turing}
\end{frame}

\section{Algorithme de Deutsch-Jozsa}
\begin{frame}
    \frametitle{Description du problème}
\end{frame}

\begin{frame}
    \frametitle{Solution classique}
\end{frame}

\begin{frame}
    \frametitle{Algorithme de Deutsch}
\end{frame}

\begin{frame}
    \frametitle{Cas général ($n$ quelconque)}
\end{frame}

\end{document}